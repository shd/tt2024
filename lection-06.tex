\documentclass[aspectratio=169,dvipsnames,usenames]{beamer}
\setbeamertemplate{navigation symbols}{}
\usepackage{graphicx}
\usepackage{stmaryrd}
\usepackage[T1,T2A]{fontenc}
\usepackage[utf8]{inputenc}
\usepackage[english,russian]{babel}
\usepackage{amsmath}
\usepackage{amsfonts}
\usepackage{amssymb}
\usepackage{makeidx}
\usepackage{verbatim}
\usepackage{amsthm}
\usepackage{bnf}
\usepackage{enumerate}
\usepackage{mathtext}
\usepackage{mathtools}
\usepackage{mathabx}
%\usepackage[left=2cm,right=2cm,top=2cm,bottom=2cm,bindingoffset=0cm]{geometry}
\usepackage{proof}
%\usepackage{paracol}
%\usepackage{enumitem}
\usepackage{xcolor}
%\usepackage{colortbl}
%\usepackage{minted}
%\usepackage{hyperref}
\usepackage{tikz}
\usetikzlibrary{calc}
\usetikzlibrary{graphs}
\usetikzlibrary{graphs.standard}
\usetikzlibrary{automata,positioning}
\usepackage{float}

\begin{document}

%\theoremstyle{dfn}
\newtheorem{dfn}{Определение}[section]
\newtheorem{nte}{Замечание}[section]

\newtheorem{axiom}{Аксиома}[section]
\newtheorem{thm}{Теорема}[section]
\newtheorem{lmm}[theorem]{Лемма}
\newtheorem{statement}{Утверждение}[section]
\newtheorem{oun_paragraph}{Пункт}[section]
\newtheorem{cons}{Следствие}[section]
\newtheorem*{exm}{Пример}

\newcommand{\comb}[1]{\operatorname{\bf{\textrm{#1}}}}
\newcommand{\func}[1]{\operatorname{#1}}
\newcommand{\reduction}[1]{{\color{OrangeRed}#1}}
\newcommand{\set}[1]{\left\{#1\right\}}

\def\from#1{\par \parbox{0.7\textwidth}{\par \hfill\raggedleft \it #1}} 

\begin{frame}{}
\begin{center}\Large Лекция 6. \\Язык Аренд \\Изоморфизм Карри-Ховарда-Воеводского.\\Гомотопическая теория типов\end{center}
\end{frame}

\begin{frame}{О языке Аренд}
\begin{itemize}
\item Разработан в Jet Brains.
\item Назван в честь Аренда Гейтинга (Arend Heyting, 9 мая 1898, Амстердам — 9 июля 1980, Лугано) --- буква <<H>> из BHK-интерпретации.
\item Язык сейчас исключительно для доказательств математических фактов: в отличие, например, от Coq ---
широко применяется для верификации софта, может использоваться для построения кода по доказательствам.
\item Основан на Гомотопической теории типов (HoTT) --- развитии интуиционистской теории типов Мартина-Лёфа.
\item Полное изучение темы мы не сможем провести --- нужен более прочный математический фундамент.
Разберём основные идеи, как введение.
\end{itemize}
\end{frame}

\begin{frame}[fragile]{Краткий перечень языковых конструкций}
\begin{itemize}
\item Синтаксис: отчасти вдохновлён ТеХ-ом.
\item Реализует конструкции из $\lambda C$: зависимые типы, пи- и сигма-типы.
\small\color[HTML]{025002}
\begin{verbatim}
\func fe : \Pi (P : Nat -> \Type) -> 
          (\Pi (a : Nat) -> (P a)) -> \Sigma (x : Nat) (P x) =>
   \lam P proof_forall => (0, proof_forall 0)
\end{verbatim}
\normalsize\color{black}
<<если утверждение $P$ истинно при всех натуральных аргументах, то существует натуральный
аргумент, при котором оно истинно>>
\item Индуктивные типы (потомки W-типов из теории типов Мартина-Лёфа) --- обобщение алгебраических.
\item Предикативная иерархия универсумов (базовые типы имеют уровень 0, тип уровня $k$ имеет уровень $k+1$).
\item Гомотопическая теория типов --- формализуется с помощью унивалентной кубической теории типов.
Равенство понимается как путь --- в отличии от системы индуктивных типов в теории типов Мартина-Лёфа.
\end{itemize}
\end{frame}

\begin{frame}[fragile]{Индуктивные типы}
\begin{itemize}
\item Алгебраические типы: параметры типа не влияют на выбор конструктора
\small\color[HTML]{025002}
\verb!type 'a list = Nil | Cons a ('a list)!
\normalsize
\color{black}

\item Обобщённые алгебраические типы: допустимые конструкторы зависят от параметров типа
\small\color[HTML]{025002}\begin{verbatim}
type _ constant =
  | Int : int -> int constant
  | Float : float -> float constant

Int 5 : int constant
Float 3.141 : float constant
\end{verbatim}\normalsize 
\color{black}

\item Обобщаем дальше (зависимые типы в параметрах, сложные конструкторы): $W$-типы, индуктивные типы в разных вариантах.
\end{itemize}
\end{frame}

%\begin{frame}{W-типы}
%$W$-типы Мартина-Лёфа, от слов well-founded.
%\end{frame}

\begin{frame}[fragile]{Индуктивные типы в Аренде}
Простой случай --- похоже на алгебраические типы:
\small\color[HTML]{025002}\begin{verbatim}
\data Nat 
   | zero
   | suc Nat
\end{verbatim}\normalsize
\color{black}

Случай сложнее:
\small\color[HTML]{025002}\begin{verbatim}
\data Fin (x : Nat) \elim x
  | suc n => { fzero | fsuc (Fin n) }

fzero : Fin 15         -- ок
fsuc (fzero) : Fin 1   -- не типизируется
\end{verbatim}
\color{black}

%\begin{verbatim}
%\data Vec (A : \Type) (x : Nat) \elim x
%  | 0 => fnil
%  | suc n => fcons A (Vec A n)
%\end{verbatim}

Аренд интересен наличием высших индуктивных типов, простой пример:

\small\color[HTML]{025002}\begin{verbatim}
\data Int
  | \coerce pos Nat
  | neg Nat \with { zero => pos zero }
\end{verbatim}\normalsize
\color{black}

\end{frame}

\begin{frame}[fragile]{Элиминаторы для индуктивных типов}
\begin{itemize}
\item Знакомы с элиминаторами начиная с алгебраических типов:
$$\infer{\Gamma \vdash \text{Case} \; L \; f \; g : \tau}{\Gamma \vdash L: \varphi \vee \psi, \;\;\;\; \Gamma \vdash f : \varphi \to \tau, \;\;\;\; \Gamma \vdash g: \psi \to \tau}$$

\item Вспомним натуральные числа:

\small\color[HTML]{025002}\begin{verbatim}
\data Nat 
   | zero
   | suc Nat
\end{verbatim}\normalsize
\color{black}

Построим зависимый элиминатор:

\small\color[HTML]{025002}\begin{verbatim}
\func Nat-elim (P : Nat -> \Type)
               (z : P zero)
               (s : \Pi (n : Nat) -> P n -> P (suc n))
               (x : Nat) : P x \elim x
  | zero => z
  | suc n => s n (Nat-elim P z s n)
\end{verbatim}\normalsize
\color{black}

\end{itemize}
\end{frame}

\begin{frame}[fragile]{Доказательство с помощью элиминаторов}
\small\color[HTML]{025002}\begin{verbatim}
\func plus-commutes-zero (x : Nat) : x Nat.+ 0 = 0 Nat.+ x \elim x
  | 0 => idp                                    -- P 0
  | suc x => pmap suc (plus-commutes-zero x)    -- P x -> P (suc x)

-- Но то же можно сделать с помощью Nat-elim:

\func Nat-elim (P : Nat -> \Type)
               (z : P zero)
               (s : \Pi (n : Nat) -> P n -> P (suc n))
               (x : Nat) : P x \elim x
  | zero => z
  | suc n => s n (Nat-elim P z s n)

\func plus-commutes-zero' (x : Nat) : x Nat.+ 0 = 0 Nat.+ x =>
  Nat-elim (\lam x => x Nat.+ 0 = 0 Nat.+ x) 
           (idp) 
           (\lam x prev => pmap suc prev) x
\end{verbatim}\normalsize
\color{black}
\end{frame}

\begin{frame}{Равенство}
%Что такое равенство? 

\begin{itemize}
\item Структурное равенство: побитовая одинаковость.
Разрешимо, но малопригодно для жизни:

$$\frac{2}{5} \ne \frac{4}{10}$$\vspace{0.5cm}

\item Доказательное равенство: на основании аксиоматики. Неразрешимо.
\end{itemize}
\end{frame}

\begin{frame}{Изоморфизм Карри-Ховарда-Воеводского}

\begin{tabular}{lll}
Логика & $\lambda$-исчисление & Топология\\\hline
Высказывание & Тип & Пространство\\
Доказательство & Терм & Точка в пространстве\\
Предикат & Зависимый тип & Расслоение\\
Равенство &\emph{Выделенный} зависимый тип & Пространство путей\\
Доказательство равенства & Элемент равенства & Путь между точками
\end{tabular}

\begin{dfn}Будем считать два значения $a$ и $b$ в пространстве $X$ равными, если они связаны путём: непрерывной функцией $f:I\rightarrow X$,
где $I = [0,1]$, такой, что $f(0)=a$, $f(1)=b$.\end{dfn}
Из определения равенства по Лейбницу следует, что если некоторый предикат истинен для левой части непрерывного пути, то он истинен и для правой части.
\end{frame}

\begin{frame}{Равенство на примере натуральных чисел}
\begin{dfn}$a,b \in \mathbb{N}_0$. Тогда $a\equiv b$, если существует непрерывная $f: I\rightarrow\mathbb{N}_0$, причём $f(0)=a$, $f(1)=b$\end{dfn}
\begin{lmm}$a\equiv b$ тогда и только тогда, когда $a=b$.\end{lmm}
\begin{proof}Пусть $a = b$. Положим $f(x) = a$. Тогда $f(0)=a=b=f(1)$.

Пусть $a \ne b$. Предположим, есть путь $f$. Рассмотрим $A=f^{-1}(a)$ и $X=f^{-1}(\mathbb{N}_0\setminus\{a\})$. 
По дискретности топологии $\{a\}$ и $\mathbb{N}_0\setminus\{a\}$ открыты. По непрерывности тогда $A$ и $X$ открыты.
Также оба непусты (очевидно, $0 \in A$, $1 \in X$), $A \cap X = \varnothing$ и $A \cup X = I$. Значит, $I$ несвязно, что не так.
Противоречие.
\end{proof}
\end{frame}

\begin{frame}{Свойства равенства}
\begin{thm}
$a = a$.
Если $a = b$, то $b = a$.
Если $a = b$ и $b = c$, то $a = c$.
\end{thm}
\begin{proof}
Положим $f(x) := a$.

Положим $g(x) := f(1 - x)$. Тогда $f(0) = b$, $f(1) = a$, композиция непрерывных --- непрерывна.

Пусть $f_1: a \leadsto b$ и $f_2: b \leadsto c$.
Тогда $$g(x) := \left\{\begin{array}{ll}f_1(2\cdot x),& x < 0.5\\f_2(2\cdot x -1),& x\ge 0.5\end{array}\right.$$
\end{proof}
\end{frame}

\begin{frame}[fragile]{Равенство в Аренде}
Компьютер дискретен, потому мы имитируем топологическую конструкцию.
\begin{itemize}
\item Интервал $\texttt{I} := [\text{left}..\text{right}]$ --- есть доступ только к граничным точкам, внутренность недоступна.
\item Путь --- индуктивный тип:
\small\color[HTML]{025002}\begin{verbatim}
\data Path (A : I -> \Type) (a : A left) (a' : A right)
  | path (\Pi (i : I) -> A i)
\end{verbatim}\normalsize
\color{black}

$\text{path f}:\text{Path}\ A\ a\ a'$  означает, что $f: a \leadsto a'$.
Заметим, что компилятор дополнительно проверяет $f\ \text{left}\twoheadrightarrow_\beta a$ и $f\ \text{right}\twoheadrightarrow_\beta a'$ --- 
неуказанная в определении встроенная семантика для \verb!path!.
\item Равенство $a = a'$ (примем $a,a':A$) --- сокращение:
\small\color[HTML]{025002}\begin{verbatim}
\func \infix 1 = {A : \Type} (a a' : A) => Path (\lam _ => A) a a'
\end{verbatim}\normalsize
\color{black}

%Кто обитает в $a = a'$?\\
$\text{path }f : a = a'$ означает, что $\text{path }f:\text{Path}\ A\ a\ a'$, то есть найдётся $f : a \leadsto a'$.
\end{itemize}
\end{frame}

\begin{frame}[fragile]{Как работать с \texttt{I}}

\small\color[HTML]{025002}\begin{verbatim}
\func \infix 1 = {A : \Type} (a a' : A) => Path (\lam _ => A) a a'
\end{verbatim}\normalsize
\color{black}

Докажем что-нибудь про равенство.
\begin{itemize}
\item Рефлексивность: 
\small\color[HTML]{025002}\verb!\func eq-reflexive {A : \Type} {a : A} : a = a => path (\lam _ => a)!\normalsize\color{black}

\item Для транзитивности нужна середина интервала. Для этого воспользуемся 
элиминатором для \verb!I!, также системное определение:
\small\color[HTML]{025002}\begin{verbatim}
\func coe (P : I -> \Type)
          (a : P left)
          (i : I) : P i \elim i
  | left => a
\end{verbatim}\normalsize
\color{black}

<<Если некоторый предикат $P$, индексированный интервалом, выполнен на левом крае интервала, то он выполнен в любой точке интервала>>

\end{itemize}
\end{frame}

\begin{frame}[fragile]{Транзитивность равенства}


\small\color[HTML]{025002}\begin{verbatim}
\func eq-transitive {A : \Type} {a a' a'' : A}
                    (f : a = a')
                    (g : a' = a'') : a = a'' =>
  coe (\lam p => a = g @ p) f right
\end{verbatim}\normalsize
\color{black}

%\vspace{-0.15cm}
%{\color{gray}\verb!\func coe (P : I -> \Type) (a : P left) (i : I) : P i \elim i!}
%\vspace{0.15cm}


%Итак: $a' = g @ left$, $a'' = g @ right$, $a_p = g @ p, p \in \texttt{I}$.

Проиллюстрируем картинкой:\vspace{-0.2cm}
\begin{center}\tikz{
   \node (A) at (0,3) { $a$ };
   \node (A1) at (3,3) { $a'$ };
   \node (A2) at (3,0) { $a''$ };
   \node[color=gray] (P) at (3,1.5) {};

   \node[right] at ($(A1) + (0.5,0)$) {\small\color{gray} $g@left$ };
   \node[right] at ($(P) + (0.5,0)$) {\small\color{gray} $g@p$ };
   \node[right] at ($(A2) + (0.5,0)$) {\small\color{gray} $g@right$ };

   \draw[-stealth] (A) -- node[above]{$f: a = a'$} (A1);
   \draw[-stealth] (A1) -- (A2);
   \draw[-stealth] (A) -- node[pos=0.6,below,sloped]{$a = a''$}(A2);
   \draw[-stealth,color=gray] (A) -- (P);
}\end{center}\vspace{-0.3cm}
Здесь $p \in \texttt{I}$, и $g @ p$ --- некоторая точка по пути $g$ из $a'$ в $a''$.

\end{frame}

\begin{frame}[fragile]{Симметричность}
Осталось третье свойство: $a = a' \rightarrow a' = a$.

\small\color[HTML]{025002}\begin{verbatim}
\func eq-symmetric {A : \Type} {a a' : A} (p : a = a') : a' = a \elim p
  | idp => idp
\end{verbatim}\normalsize
\color{black}

Сопоставление с образцом для равенства возможно --- 
его обитатель есть \verb!idp!, специальный конструктор. Также встроенное в язык поведение.

%В гомотопической теории типов эта конструкция выражается через так называемый J-элиминатор:

\end{frame}

\begin{frame}[fragile]{Стандартные функции работы с равенством}
\begin{itemize}
\item Если $a = a'$ и $B(a)$ истинен, то $B(a')$ истинен.

\small\color[HTML]{025002}\begin{verbatim}\func transport {A : \Type} (B : A -> \Type) {a a' : A} 
              (p : a = a') (b : B a) : B a'
     => coe (\lam i => B (p @ i)) b right\end{verbatim}
\normalsize\color{black}

По сути тот же \verb!coe!, но индексирует предикат вдоль пути вместо интервала:

\small\color[HTML]{025002}\begin{verbatim}\func eq-transitive {A : \Type} {a a' a'' : A}
                    (f : a = a')
                    (g : a' = a'') : a = a'' =>
  transport (\lam x => a = x) f right
\end{verbatim}\normalsize
\color{black}

\item Если $a = a'$, то $f(a) = f(a')$:

\small\color[HTML]{025002}\begin{verbatim}
\func pmap {A B : \Type} (f : A -> B) {a a' : A} (p : a = a') : f a = f a' 
     => path (\lam i => f (p @ i))
\end{verbatim}\normalsize
\color{black}
\end{itemize}

\end{frame}

\begin{frame}[fragile]{Функциональная экстенциональность}
Всюду совпадающие функции равны.

\small\color[HTML]{025002}\begin{verbatim}\func funExt {A : \Type} (B : A -> \Type) {f g : \Pi (a : A) -> B a}
    (p : \Pi (a : A) -> f a = g a) : f = g
    => path (\lam i => \lam a => p a @ i)\end{verbatim}
\normalsize
\color{black}
\end{frame}

\begin{frame}[fragile]{Доказать неравенство}
\begin{itemize}
\item Ложь --- это \verb!Empty! (тип без значений).
\small\color[HTML]{025002}
\begin{verbatim}
\data Empty
\end{verbatim}\normalsize
\color{black}
\item $0 \ne 1$ --- это $0 = 1 \rightarrow \bot$.
\item Построим функцию \verb!T : Nat -> \Type!, которая для 0 возвращает обитаемый тип, а для других чисел --- необитаемый.
\small\color[HTML]{025002}\begin{verbatim}
\func T (x : Nat) : \Type \elim x
| Z => \Sigma
| _ => Empty
\end{verbatim}\normalsize
\color{black}
\item $T$ --- это предикат, раз можем поставить вопрос, истинен (обитаем) ли $T\ 0$. Поэтому он сохраняет истинность для равных объектов
(принцип Лейбница).
\item Заметим, что \verb!T 0! истинен (обитаем), так как \verb!() : \Sigma!. Значит, мы найдём значение и для \verb!T 1! (для \verb!Empty!), при условии, что $0 = 1$.
\small\color[HTML]{025002}\begin{verbatim}
zero-ne-1 (p : 0 = 1) : Empty => transport T p ()
\end{verbatim}\normalsize
\color{black}
\end{itemize}
\end{frame}

\end{document}
