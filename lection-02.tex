\documentclass[aspectratio=169]{beamer}
\setbeamertemplate{navigation symbols}{}
\usepackage{graphicx}
\usepackage{stmaryrd}
\usepackage[T1,T2A]{fontenc}
\usepackage[utf8]{inputenc}
\usepackage[english,russian]{babel}
\usepackage{amsmath}
\usepackage{amsfonts}
\usepackage{amssymb}
\usepackage{makeidx}
\usepackage{verbatim}
\usepackage{amsthm}
\usepackage{bnf}
\usepackage{tikz}
\usepackage{enumerate}
\usepackage{mathtext}
\usepackage{mathtools}
\usepackage{mathabx}
%\usepackage[left=2cm,right=2cm,top=2cm,bottom=2cm,bindingoffset=0cm]{geometry}
\usepackage{proof}
%\usepackage{paracol}
%\usepackage{enumitem}
\usepackage{color}
\usepackage{colortbl}
\usepackage{minted}
%\usepackage{hyperref}
\usetikzlibrary{graphs}
\usetikzlibrary{graphs.standard}
\usetikzlibrary{automata,positioning}
\usepackage{float}

\begin{document}

%\theoremstyle{dfn}
\newtheorem{dfn}{Определение}[section]
\newtheorem{nte}{Замечание}[section]

\newtheorem{axiom}{Аксиома}[section]
\newtheorem{thm}{Теорема}[section]
\newtheorem{lmm}[theorem]{Лемма}
\newtheorem{statement}{Утверждение}[section]
\newtheorem{oun_paragraph}{Пункт}[section]
\newtheorem{cons}{Следствие}[section]
\newtheorem*{exm}{Пример}

\newcommand{\comb}[1]{\operatorname{\bf{\textrm{#1}}}}
\newcommand{\func}[1]{\operatorname{#1}}
\newcommand{\reduction}[1]{{\color{OrangeRed}#1}}
\newcommand{\set}[1]{\left\{#1\right\}}

\def\from#1{\par \parbox{0.7\textwidth}{\par \hfill\raggedleft \it #1}} 

\begin{frame}{}
\begin{center}\Large Лекция 2.\\ \Large Задачи типизации $\lambda_\rightarrow$. \\Выразительная сила $\lambda_\rightarrow$.\end{center}
\end{frame}

\begin{frame}[fragile]{Основные задачи типизации $\lambda$-исчисления}

Рассмотрим $? \vdash ? : ?$.

		\begin{enumerate}
			\item \emph{Проверка типа:} выполняется ли $\Gamma\vdash M:\sigma$ для контекста $\Gamma\text{, терма }M\text{ и типа }\sigma$\\

{\color{gray}Компиляция в языке программирования, типизированном по Чёрчу. Проверка доказательства.}
			\item \emph{Реконструкция типа:} $?\vdash M:\:?$.

{\color{gray}Компиляция в языке программирования, типизированном по Карри. Это бывает чаще, чем кажется.}

\verb!template <class A, class B>!\\
\verb!auto min(A a, B b) -> decltype(a < b ? a : b) {!\\
\verb!    return (a < b) ? a : b;!\\
\verb!}!

			\item \emph{Обитаемость типа:} $\Gamma\vdash ?:\sigma$.

{\color{gray}Поиск доказательства.}
		\end{enumerate}
Все задачи разрешимы.
\end{frame}

\begin{frame}{Задача реконструкции типа}
\begin{dfn}Алгебраический терм $$\theta ::= x\:|\:(f\:\theta\:\ldots\:\theta)$$\end{dfn}\vspace{-0.3cm}
\begin{dfn}Подстановка переменных --- функция $S_0: V \rightarrow T$, где $S_0(x) = x$ почти везде
(за исключением конечного множества переменных).

Подстановка: $S: T \rightarrow T$, что $S(x) = S_0(x)$, но $S(f\ \theta_1\ \dots \theta_k) = f\ S(\theta_1)\ \dots\ S(\theta_k)$
$S(\Gamma) = \{ x : S(\tau_x)\ |\ x : \tau_x \in \Gamma\}$
\end{dfn}

\begin{dfn}Будем воспринимать запись типа как некоторое выражение в алгебраических термах, импликация --- единственный
функциональный символ.
Наиболее общей парой для задачи реконструкции типа $?\vdash M:?$ назовём такие $\langle \Gamma, \gamma \rangle$, что:
\begin{enumerate}
\item $\Gamma\vdash M:\gamma$
\item Если $\Delta\vdash M:\delta$, то найдётся такая подстановка $S$, что $\Delta = S(\Gamma)$ и $\delta = S(\gamma)$.
\end{enumerate}
\end{dfn}

\end{frame}

\begin{frame}{Общий план решения}
\begin{enumerate}
\item Основа решения --- алгоритм унификации для системы уравнений в алгебраических термах.
\item По терму $M$ строим систему уравнений в алгебраических термах.
\item Наиболее общим унификатором системы будет является подстановка, из которой можно получить наиболее общую пару.
\end{enumerate}
\end{frame}

\begin{frame}{Система уравнений в алгебраических термах}
	\begin{dfn}Система уравнений в алгебраических термах\end{dfn}
	$$
		\begin{cases}
			\theta_1=\sigma_1&\\
			\vdots&\\
			\theta_n=\sigma_n&\\
		\end{cases}
	$$\par где $\theta_i \text{ и } \sigma_i-\text{термы}$\par
\end{frame}

\begin{frame}{Задача унификации}

\begin{dfn}Решением задачи унификации для системы уравнений $\sigma_k = \tau_k$ назовём такую подстановку
$S$, что $S(\sigma_k) = S(\tau_k)$.
\end{dfn}
\begin{dfn}Наиболее общим решением задачи унификации назовём такую подстановку $S$, что для любого
другого решения $T$ найдётся подстановка $R$, что $T(\rho) = R(S(\rho))$.
\end{dfn}
\begin{dfn}Система в разрешённой форме --- каждое уравнение имеет вид $x_i = \theta_i$, причём
каждый из $x_i$ входит в систему ровно один раз (является левой частью одного из уравнений)
\end{dfn}
\begin{dfn}Система несовместна --- система не имеет решений.\end{dfn}
\end{frame}

\begin{frame}{Алгоритм унификации}
Пусть дана система уравнений $\sigma_i = \tau_i$. 
Возьмём произвольное уравнение и попробуем проверить/применить одно из следующих условий/действий к нему:

		\begin{enumerate}[(a)]
		\item $\sigma_i=x$ если $\sigma_i$ не переменная перепишем как $x=\sigma_i$
		\item $\sigma_i = \sigma_i$ удалим
		\item $f\:\theta_1\hdots\theta_n=f\:\rho_1\hdots\rho_n$ заменим на $n$ уравнений $\theta_k = \rho_k$
		\item если уравнение имеет вид $x=\tau_i$ и $x$ входит хотя бы в одно другое уравнение, то заменим все другие уравнения на $\sigma_k[x := \tau_i] = \tau_k[x := \tau_i]$
		\item если уравнение имеет вид $x = f\ \dots x_i \dots$, система несовместна (occurrs check)
		\item если уравнение имеет вид $f\ ... = g\ ...$ при $f \ne g$, система несовместна.
		\end{enumerate}

Если нет ни одного подходящего правила ни для одного уравнения --- закончим работу (система находится в разрешённой форме).
\end{frame}

\begin{frame}{Алгоритм всегда завершает работу}
\begin{itemize}
\item Рассмотрим $\langle x,y,z \rangle$,
где:
\begin{itemize}\item $x$ --- количество переменных, входящих в систему, которые входят не в разрешённом виде.
Переменная $t$ входит в систему в разрешённом виде, если переменная входит в систему ровно один раз, причём входит в уравнение вида $t = \sigma$;
\item $y$ --- количество функциональных символов в системе;
\item $z$ --- количество уравнений типа $a=a$ и $\theta=b$, где $\theta$ не переменная.
\end{itemize}
\item Упорядочим тройки лексикографически (согласно порядковому типу $\omega^3$).

\item Заметим, что операции $(a)$ и $(b)$ всегда уменьшают $z$ и иногда уменьшают $x$.
Операция $(c)$ всегда уменьшает $y$, иногда $x$ и, возможно, увеличивает $z$.
Операция $(d)$ всегда уменьшает $x$ и иногда увеличивает $y$. То есть, операции $(a)-(d)$ всегда уменьшают
соответствующий ординал.

\item Cогласно лемме из матлога любая строго убывающая последовательность ординалов имеет конечную длину.
\end{itemize}
\end{frame}

\begin{frame}{Корректность алгоритма}
\begin{thm}Для системы уравнений $\sigma_k = \tau_k$ алгоритм даёт наиболее общее решение, если оно существует.\end{thm}
\begin{proof}
\begin{itemize}
\item Операции $(a)-(d)$ не меняют множества решений системы. За конечное время либо выполнится условие $(e)$ или $(f)$,
либо будут исчерпаны правила.
\item Условия $(e)$, $(f)$ очевидно означают несовместность системы (в т.ч. исходной).
\item При отсутствии возможности применения правил и условий все уравнения имеют вид $x = \theta_x$,
где $x$ входит в систему только один раз. Построим $S_0(x) = \theta_x$. 

\item Если есть подстановка $T: T(\sigma_k) = T(\tau_k)$, тогда положим $R = \mathcal{U}(\{ S_0(x) = T_0(x)\ |\ x \ne T_0(x)\})$.
Очевидно, $T(\zeta) = R(S(\zeta))$
\end{itemize}
\end{proof}
\end{frame}

\begin{frame}{Построение системы по терму $M$}
Будем строить систему рекурсией по структуре терма $M$ (предполагаем, что все имена
для связанных переменных уникальны). 
Каждой переменной $x$ сопоставим свежую типовую переменную $\alpha_x$. Также каждой аппликации $P\ Q$ в терме сопоставим
свежую типовую переменную $\beta_{PQ}$. 


%В итоге получим $\langle \mathcal{E}, \sigma\rangle$, где $\mathcal{E}$ ---
%система уравнений, а $\sigma$ --- тип терма $M$. 
По терму $M$ и по всем его подтермам рекурсивно построим пару $\langle \mathcal{E}_M, \sigma_M\rangle$ так:

%Учитывайте недостаток записи: здесь две разные переменные $\beta_{xx}$: $(x\ x)\ (x\ x)$

$$\langle\mathcal{E}_M,\sigma_M\rangle := \left\{\begin{array}{ll}\langle \varnothing, \alpha_x\rangle, & M = x\\
\langle \mathcal{E}_P, \alpha_x\rightarrow\sigma_P\rangle, & M = \lambda x.P\\
\langle \mathcal{E}_P\cup\mathcal{E}_Q\cup\{\sigma_P = \sigma_Q\rightarrow\beta_{PQ}\}, \beta_{PQ}\rangle, 
   & M = P\ Q
\end{array}\right.$$

\begin{thm} Если $S = \mathcal{U}(\mathcal{E}_M)$, то наиболее общим решением задачи типизации будет
$\langle\{ x : S(\alpha_x)\ |\ x \in FV(M) \}, S(\sigma_M)\rangle$\end{thm}
\begin{proof} Индукция по структуре $M$. \end{proof}
\end{frame}

\begin{frame}{Пример вывода типов }
\begin{enumerate}
\item Выберем пример ($M = \lambda f.\lambda x.f\ (f\ x)$) и индуктивно составим систему:
\begin{itemize}
\item Для $f\ x$:\quad $\langle\{\alpha_f = \alpha_x\rightarrow\beta_{fx}\},\; \beta_{fx}\rangle$
\item Для $f\ (f\ x)$:\quad $\langle\{\alpha_f = \alpha_x\rightarrow\beta_{fx}, \alpha_f = \beta_{fx}\rightarrow\beta_{ffx}\}, \beta_{ffx}\rangle$
\item Для $\lambda x.f\ (f\ x)$:\quad $\langle\{\alpha_f = \alpha_x\rightarrow\beta_{fx}, \alpha_f = \beta_{fx}\rightarrow\beta_{ffx}\}, \alpha_x\rightarrow\beta_{ffx}\rangle$
\item Для $\lambda f.\lambda x.f\ (f\ x)$:\quad $\langle\{\alpha_f = \alpha_x\rightarrow\beta_{fx}, \alpha_f = \beta_{fx}\rightarrow\beta_{ffx}\}, \alpha_f\rightarrow\alpha_x\rightarrow\beta_{ffx}\rangle$
\end{itemize}
\item Приводим систему к разрешённой форме:
\begin{itemize}
\item $\alpha_f = \alpha_x\rightarrow\beta_{fx}, \alpha_f = \beta_{fx}\rightarrow\beta_{ffx}$
\item $\alpha_f = \alpha_x\rightarrow\beta_{fx}, \alpha_x\rightarrow\beta_{fx} = \beta_{fx}\rightarrow\beta_{ffx}$, правило $(d)$
\item $\alpha_f = \alpha_x\rightarrow\beta_{fx}, \alpha_x = \beta_{fx}, \beta_{fx} = \beta_{ffx}$, правило $(c)$
\item $\alpha_f = \beta_{fx}\rightarrow\beta_{fx}, \alpha_x = \beta_{fx}, \beta_{fx} = \beta_{ffx}$, правило $(d)$
\item $\alpha_f = \beta_{ffx}\rightarrow\beta_{ffx}, \alpha_x = \beta_{ffx}, \beta_{fx} = \beta_{ffx}$, правило $(d)$
\end{itemize}
\item Строим функцию подстановки:
$S_0(\alpha_f) = \beta_{ffx}\rightarrow\beta_{ffx}, S_0(\alpha_x) = S_0(\beta_{fx}) = \beta_{ffx}$
\item Наиболее общая пара:
$\langle \varnothing, S(\alpha_f\rightarrow\alpha_x\rightarrow\beta_{ffx})\rangle$, то есть
$$\vdash \lambda f.\lambda x.f\ (f\ x) : (\beta_{ffx}\rightarrow\beta_{ffx})\rightarrow\beta_{ffx}\rightarrow\beta_{ffx}$$
\end{enumerate}
\end{frame}

\begin{frame}{Проверка типа}
\begin{enumerate}
\item Задача реконструкции типа находит наиболее общую типизацию.
\item Сведём задачу проверки $\Gamma\vdash M:\sigma$ к задаче реконструкции типа $?\vdash M:?$ и найдём $\langle \Gamma', \sigma' \rangle$.
\item Проверим, является ли $\langle \Gamma, \sigma\rangle$ частным случаем $\langle \Gamma', \sigma' \rangle$.
\end{enumerate}
\end{frame}

\begin{frame}{Обитаемость типа}
\begin{enumerate}
\item Задача поиска $M$, что $\Gamma\vdash M:\sigma$.
\item Эквивалентно поиску доказательства утверждения $\sigma$ в ИИП (разрешимо).
\item По доказательству затем получим его краткую запись в виде терма.
\end{enumerate}
\end{frame}

\begin{frame}{Выразительная сила}
\begin{dfn}
Расширенный полином, где $P(x)$, $P(x,y)$ --- полиномы (выражения, составленные из сложения, умножения, аргументов и натуральных констант),
$c$ --- константа:
$$E(m,n) := \left\{\begin{array}{ll}
c,& m = 0, n = 0\\
P_1(m), & n=0\\
P_2(n), & m=0\\
P_3(m,n), & m>0, n>0
\end{array}\right.$$
\end{dfn}

\begin{thm}
Пусть $\eta = (\alpha\rightarrow\alpha)\rightarrow(\alpha\rightarrow\alpha)$. 
Если $F: \eta\rightarrow\eta\rightarrow\eta$, то найдётся такой расширенный
полином $E(m,n)$, что при всех $m,n \in \mathbb{N}_0$ выполнено $F\ \overline{m}\ \overline{n} =_\beta \overline{E(m,n)}$, 
либо $F\ \overline{m}\ \overline{n} =_\beta \lambda f.f$ при $E(m,n) = 1$.
\end{thm}
\end{frame}

\begin{frame}{Расширение языка: полное ИИВ}
\begin{itemize}
\item Попробуем увеличить выразительную силу, воспользовавшись изоморфизмом Карри-Ховарда. Рассмотрим полное ИИВ.
\item Расширим язык: $$\begin{array}{rcll}\Lambda &::=& x\ |\ (\Lambda\ \Lambda)\ |\ (\lambda x.\Lambda)\ \\
& | & \langle\Lambda,\Lambda\rangle\ |\ (\pi_L\ \Lambda)\ |\ (\pi_R\ \Lambda) & \text{термы для }\&\\
& | & (In_L\ \Lambda)\ |\ (In_R\ \Lambda)\ |\ (Case\ \Lambda\ \Lambda\ \Lambda) & \text{термы для }\vee\\
& | &(Absurd\ \Lambda) & \text{термы для }\bot
\end{array}$$
\end{itemize}

\end{frame}

\begin{frame}{Новые связки требуют отдельных правил}
\begin{itemize}\item Упорядоченная пара в бестиповом лямбда-исчислении.
$$\text{MkPair} ::= \lambda a.\lambda b.\underbrace{\left(\lambda p.p\ a\ b\right)}_{\text{MkPair a b}}
\quad\quad\text{Fst} ::= \lambda p.p\ \text{T}
\quad\quad\text{Snd} ::= \lambda p.p\ \text{F}$$
\item Какой тип у $\text{MkPair a b}$?

$$\text{MkPair a b} = \lambda p^{\alpha\rightarrow\beta\rightarrow\gamma}.p\ a^{\alpha}\ b^{\beta} : (\alpha\rightarrow\beta\rightarrow\gamma)\rightarrow\gamma$$

\item Тип зависит от типа результата $\gamma$: при левой проекции $\alpha = \gamma$ $$\text{Fst} = \lambda p.p^{(\alpha\rightarrow\beta\rightarrow\gamma)\rightarrow\gamma}\ \text{T}^{\alpha\rightarrow\beta\rightarrow\alpha} : \gamma$$

При правой проекции $\beta = \gamma$: $\text{Snd} = \lambda p.p^{(\alpha\rightarrow\beta\rightarrow\gamma)\rightarrow\gamma}\ \text{F}^{\alpha\rightarrow\beta\rightarrow\beta} : \gamma$

\item Как известно, связки в ИИВ друг через друга не выражаются. Поэтому никакая формула не сможет типизировать упорядоченную пару. 
Однако, в данном варианте типизации может помочь квантор по $\gamma$ или схема аксиом (правил вывода).
\end{itemize}
\end{frame}

\begin{frame}{Дополнительные правила для расширенного языка}
\begin{enumerate}
	\item Типизация дизъюнкции (алгебраического типа)\[
	\infer{\Gamma \vdash In_L \; A: \varphi \vee \psi}{\Gamma\vdash A: \varphi}
	\quad\quad
	\infer{\Gamma \vdash In_R \; B: \varphi \vee \psi}{\Gamma\vdash B: \psi}
	\]\vspace{-0.5cm}
	\[
	%\quad\quad
	\infer{\Gamma \vdash \text{Case} \; L \; f \; g : \tau}{\Gamma \vdash L: \varphi \vee \psi, \;\;\;\; \Gamma \vdash f : \varphi \to \tau, \;\;\;\; \Gamma \vdash g: \psi \to \tau}
	\]

	\item Типизация конъюнкции (упорядоченной пары)
	\[
	\infer{\Gamma \vdash \langle A, B \rangle : \varphi \text{\&} \psi}{\Gamma\vdash A: \varphi && \Gamma\vdash B: \psi}
	%\]\vspace{-0.5cm}
	%\[
	\quad\quad
	\infer{\Gamma \vdash \text{Fst}\ P : \varphi}{\Gamma \vdash P: \varphi \text{\&} \psi}
	\quad\quad
	\infer{\Gamma \vdash \text{Snd}\ P : \psi}{\Gamma \vdash P: \varphi \text{\&} \psi}
	\]

	\item Типизация лжи
	\[
	\infer{\Gamma \vdash \text{Absurd}\ A : \varphi}{\Gamma \vdash A : \bot}
	\]
\end{enumerate}
\end{frame}

\end{document}
